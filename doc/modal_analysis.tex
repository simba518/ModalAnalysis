\documentclass[9pt,twocolumn]{extarticle}

\usepackage[hmargin=0.5in,tmargin=0.5in]{geometry}
\usepackage{amsmath,amssymb}
\usepackage{times}
\usepackage{graphicx}
\usepackage{subfigure}

\usepackage{cleveref}
\usepackage{color}
\newcommand{\TODO}[1]{\textcolor{red}{#1}}

\newcommand{\FPP}[2]{\frac{\partial #1}{\partial #2}}
\newcommand{\argmin}{\operatornamewithlimits{arg\ min}}
\author{Siwang Li}

\title{Modal Analysis}

%% document begin here
\begin{document}
\maketitle

\setlength{\parskip}{0.5ex}

\section{Introduction}
Given a stiffness matrix $K$ and mass matrix $M$, we solve for the following general eigenvalue problem for eigenvectors $w$ and eigenvalues $\lambda$
\begin{equation} \label{ma}
  Kw = \lambda Mw
\end{equation}
Usually we truncate the first few eigenvalues $\Lambda = (\cdots, \lambda_i, \cdots)$ and eigenvectors $W = (\cdots, w_i, \cdots)$.

\section{Hex-mesh and Tet-mesh}
We can use either Tet-mesh or Hex-mesh to generate $K$ and $M$. And for the same number of nodes, we will obtain different $K, M$ by using different type of mesh, and thus different $\Lambda$ and $W$. We define the modal analysis introduced above as a function: 
\begin{equation} \label{ma-f}
  (\Lambda,W) = F_{ma}(M,N)
\end{equation}
where $M$ is the type of mesh which can be either Hex-mesh or Tet-mesh, while $N$ is the number of nodes. \TODO{Suppose that, when we increase $N$, the resulting $\Lambda$ will monotonic decrease, and $(\Lambda,W)$ will convergent to some ground truth values (But why?).} 


\end{document}